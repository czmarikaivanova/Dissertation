
%
% CUSTOM THEOREMS (MI)
%
\newtheorem{theorem}{\textbf{Theorem}}
%\newtheorem{observation}[theorem]{\textbf{Observation}}
%\newtheorem{definition}{\textbf{Definition}}
%\newtheorem{proposition}[theorem]{\textbf{Proposition}}
\newtheorem{remark}[theorem]{\textbf{Remark}}
%\newtheorem{lemma}[theorem]{\textbf{Lemma}}

\chapter{Introduction}
%
This dissertation is based on six research papers that are focused on three different topics summarized in chapters of the thesis.

The first three papers concern NP-hard problems arising in ad-hoc wireless communication summarized in Chapter 2.
In general, the task is to broadcast a message in a given network of wireless devices while minimizing the power consumption.
Problems in this category differ in requirements on the network connectivity, models of power consumption, and the ability of devices to initiate a signal transmission
Some of the common features of these problems are that a device can transmit a signal to all devices within its communication vicinity at the same time,
and that a signal can travel from its originator to its recipient via multiple intermediate devices.
The wireless networks are modeled by means of graph theory.
Solution techniques for these problems involve mainly methods of integer linear programming and inexact algorithms with or without performance guarantee.

The next paper is focused on the problem of minimum broadcast time.
Unlike the previous topic, devices in this problem are supposed to send a signal to up to one neighbouring device at a time.
The objective is to determine a sequence of signal transmission from a given set of source devices to the remaining ones while minimizing the time needed for spreading the signal.
Chapter 3 describes this problem in detail along with several related problems.
This problem is also studied from the integer linear programming perspective as well as inexact algorithm perspectivea
Continuous relaxations of ILP models help to evaluate quality of a studied inexact method.
The stronger model, the better assessment it provides.

The last two paers are dedicated to problems belonging to path planning for multiple robots discussed in Chapter 4.
In general, these problems involve a group of agents (robots) initially deployed in an environment, and the task is to find a sequence of their 
so that they reach some pre-defined destination locations while optimazing some criteria such as minimum makespan or minimum total arrival time.
The agents' movement must obey a set of given rules.
An extension of the problem considers agents divided into two (or more) adversarial teams, where the teams can have either symmetric or asymmetric objectives.
After introducing the adversarial element, the problem becomes PSPACE-hard, like many two player games with alternating turns.
%Additional requirement of preserving a communication possibility during the agets' movement has also been studied.

%The remainder of this chapter is structured as follows: 
%Section~\ref{sec:back} contains an overview of general underlying mathematical and algorithmical tools and methods employed in the attached papers.
%In Section~\ref{sec:smt}, we discuss ad-hoc wireless networks and describe several optimization problems in this area.
%In particular, we focus on the shared multicast tree problem and provide an insigt into its relation to other similar problems.
%Section~\ref{sec:mbt} introduces the minimum broadcast tree problem and discusses other relevant problems.
%In Section~\ref{sec:app}, we describe the problems in multi-robot path planning, and summarize their properties.
%An emphasis is given to the area protection problem.

%This is the introduction~\cite{deb01, pyprop}\ldots


