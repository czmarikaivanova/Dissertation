\chapter{Multi-Agent Path Finidng}\label{sec:app}

In multi-robot path finding (MPF) we consider an environment with several identical moving entities referred to as 
\emph{agents}. Source and target locations are uniquely determined for each agent. The objective
is to find a route for every agent from its source to its target.
Agents must not collide with obstacles and other agents that are also moving along
planned routes towards their own targets.
The environment is modeled as an undirected graph, where the agents are placed in the vertices, and move along the edges from one vertex to another. 
Continuous time is divided into discrete time steps, where the relocation of an agent from one vertex to its neighbor takes exactly 1 time step.

Different variants and restrictions have been studied.

\emph{Muliti-agent path-finding} (MPF) \cite{source1}  supposes a group of agents in a given
environment, where initial and target positions are determined for each agent.
All individual agents must avoid collisions. Movement of the agents is carried
out in discrete time steps. Agent can shift from one vertex to its neighbor on
condition that the neighbor is either unoccupied or is being left by other agent
in the same time step. At most one agent is allowed to pass an edge within one
time step. That is, agents are not allowed to exchange their positions within one
time step.

\emph{Pebble motion on graphs} \cite{source1,source2} is a very similar problem as MPF. It can
be regarded as a restricted variant of MPF. The difference consists in rules for
movement. While MPF enables entering a vertex that is simultaneously being
left by other agent, such transfer is not permissible in pebble motion. As an
illustration we can mention 15 puzzle also known as Lloyd’s 15 \cite{lloydXX}.

\emph{Cooperative Path-finding} \cite{cpf} is a special case of MPF where each agent is
assumed to have a full knowledge of all other agents and their planned routes.
Precisely speaking, solving algorithm can take into account paths planned for
agents that were processed earlier, and adjust paths searched later according to
them.

\section{Scenarios with adversaries}

\subsection{Adversarial Cooperative Path Finding}

\subsection{Area Protection Problem}

\subsection{Area Protection Problem with Communication maintenance}

